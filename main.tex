\documentclass{amsart}
\headheight=6.15pt \textheight=8.75in \textwidth=6.5in
\oddsidemargin=0in \evensidemargin=0in \topmargin=0in

\title[Problems in Geometric Flows and Solitons]{Problems in Geometric Flows and Solitons}
\author[]{Hojoo Lee} 
\author[]{Peter Wheeler} 
\author[]{around 8 more for initial draft}

\theoremstyle{definition}
\newtheorem{Thm}{Theorem}[section]
\newtheorem{Que}{Question}[section]
\newtheorem{Prob}{Problem}[section]
\newtheorem{Con}{Conjecture}[section]
\newtheorem{Cor}{Corollary}[section]
\newtheorem{Lem}{Lemma}[section] 
\newtheorem{Def}{Definition}[section] 
\newtheorem{Exa}{Example}[section] 
\newtheorem{Rem}{Remark}[section] 
\newtheorem{Prop}{Proposition}[section] 

\usepackage{epsfig,color}
\usepackage{hyperref}

\usepackage[style=authoryear,backend=biber]{biblatex}
\bibliography{refs}
\AtEveryBibitem{\clearfield{url}}
\AtEveryBibitem{\clearfield{issn}}

\begin{document}
\maketitle
\tableofcontents

\section{Introduction}
\label{sec:intro}
 
 \newpage
 
\section{Overture} 

\begin{enumerate}
\item[]
\item[] \textbf{\textcolor{black}{Notice for the arXiv version (updated on 15 May, 2017.)}}
\item[]
\item Part of this work will be edited into an article, which will be submitted to arXiv.
\item The tentative working title is \textbf{Open problems in geometric analysis}. 
\item {\textcolor{red}{The initiator of new section will be included as a co-author of the arXiv article.}} 
\item The list of contributors will be listed in the Acknowledgement.
\item {\textcolor{red}{Our target date to submit part of this work to math.DG is  \textbf{15 September 2017.}}} 
\item[]
\end{enumerate}
\begin{enumerate}
\item[] \textbf{\textcolor{black}{Instruction for Contributors (updated on 15 May, 2017.)}}
\item[]
\item {\textcolor{red}{The tentative deadline to initiate new section is \textbf{15 August 2017.}}}
\item Initiators of new sections are assumed to play a role of the editor of their sections.
\item \textcolor{red}{When contributors add known or improved results, please provide the \textbf{references}.} 
\item When you initiate a new section, please provide a background within three pages. 
\item  {\textcolor{red}{It is prohibited to add your own latex enviroment (like using \textit{newcommand}).}} 
\item Each section has the list of references at its end. Each item reveals the family name of the coauthors and the year of the publication.
For instance, use \textbf{Perelman 2002} to indicate his arXiv preprint \emph{The entropy formula for the Ricci flow and its geometric applications}.
\item[]
\item[] \textbf{\textcolor{black}{Editorial Board (updated on 15 May, 2017.)}}
\item[]
\item[] Paul Bryan  (\href{your mail}{your mail}) 
\item[] Hojoo Lee  (\href{momentmaplee@gmail.com}{momentmaplee@gmail.com}) 
\item[] Glen Wheeler  (\href{your mail}{your mail}) 
\item[]
\item[] Please do not hesitate to contact us if you have any comments, questions, and suggestions. 
\end{enumerate}

\section{Introduction to extrinsic curvature flows}
  
\newpage 

\section{Mean curvature flow}
\begin{refsection}

\begin{itemize}
\item Estimation of the position of the extinction point -- see Bryant-Griffiths for a good discussion. Proved that there does not exist a formula from initial data, but a sharp estimate could be possible. An idea for this could be something related to the centre of mass.
\end{itemize}

\begin{Thm}[\cite{GageHamilton1986}]
Under curve shortening flow, every simple closed convex curve in ${\mathbb{R}}^{2}$ remains convex and eventually becomes extinct in a round point.
\end{Thm}

\begin{Thm}[\cite{White1995}]
Let ${M}_{t}$ be a weak MCF in ${\mathbb{R}}^{2}$ that starts from a closed surface ${M}_{0}$ of genus $g_0$. Then at each time $t > 0$ where ${M}_{t}$ is smooth, the genus of  ${M}_{t}$ is at most $g_0$.
\end{Thm}

\begin{Thm}[\cite{Huisken1984}]
Under MCF, every closed convex hypersurface in ${\mathbb{R}}^{n+1 \geq 3}$ remains convex and eventually becomes extinct in a round point.
\end{Thm}
 
\begin{Thm}[\cite{White2000}]
The singular set of a compact mean convex MCF in ${\mathbb{R}}^{n+1}$ has parabolic Hausdorff dimension at most $n-1$.
\end{Thm}

\begin{Thm}[\cite{ColdingMinicozzi2012}]
In ${\mathbb{R}}^{n+1}$ the round hypersphere ${\mathbb{S}}^{n}$ is the only closed smooth $F$-stable shrinker.
\end{Thm}

\begin{Thm}[\cite{ColdingIlmanenMinicozziWhite2013}] \label{Colding Ilmanen Minicozzi White 2013}
Given $n$, there exists  $\varepsilon = \varepsilon(n) > 0$ so that if $\Sigma \subset {\mathbb{R}}^{n+1}$
is a closed shrinker not equal to the round hypersphere, then $\lambda\left(\Sigma\right) \geq \lambda\left({\mathbb{S}}^{n}\right) + \varepsilon$. Moreover,
if $\lambda\left(\Sigma\right) \leq min \left\{  \lambda \left({\mathbb{S}}^{n-1}\right) , 3 \right\}$, then $\Sigma$ is diffeomorphic to ${\mathbb{S}}^{n}$.
\end{Thm}
 
 \begin{Thm}[\cite{BernsteinWang2016}]
Theorem \ref{Colding Ilmanen Minicozzi White 2013} holds  with $\varepsilon = 0$ for any closed hypersurface.
\end{Thm}

\begin{Con}[\cite{ColdingMinicozziPedersen2015}] Theorem \ref{Colding Ilmanen Minicozzi White 2013} holds for any
non-flat shrinker ${\Sigma}^{n}  \subset {\mathbb{R}}^{n+1}$ with $n \leq 6$.
\end{Con}

\begin{Thm}[\textbf{Rigidity of cylinders}, \cite{ColdingIlmanenMinicozzi2015}]
If a singular point of an MCF is cylindrical, then for every tangent flow there is a multiplicity one cylinder. 
\end{Thm}

\begin{Con}[\cite{ColdingMinicozziPedersen2015}] Let ${M}_{t}$  be an MCF flow of smooth closed hypersurfaces in
${\mathbb{R}}^{n+1}$. If the flow has a cylindrical singularity at time t0 and at the point $x_0 \in {\mathbb{R}}^{n+1}$, then in an 
entire space-time neighborhood of $\left(x_0,t_0\right)$ the evolving hypersurfaces have positive mean curvature.
\end{Con}

\begin{Thm}[\textbf{Conical asymptotic rigidity}, \cite{Wang2014}]
If ${\Sigma}_{1}$ and ${\Sigma}_{2}$ are shrinkers in ${\mathbb{R}}^{n+1} \ B_{R}$ that have boundary in $\partial B_{R}$ 
and are asymptotic to the same cone, then they coincide.
\end{Thm}

 \begin{Thm}[\cite{ColdingMinicozzi2015a}] Let ${M}_{t}$ be an MCF in ${\mathbb{R}}^{n+1}$. At each cylindrical singular
 point the tangent flow is unique. That is, any other tangent flow is also a cylinder with the same ${\mathbb{R}}^{k}$ factor 
 that points in the same direction.
\end{Thm}
   
\begin{Con}[\cite{ColdingMinicozzi2016}]
  Let ${M}_{t}$ be an MCF of closed embedded hypersurfaces in ${\mathbb{R}}^{n+1}$ with only cylindrical singularities. Then 
  the space-time singular set has only finitely many components.
\end{Con}

\begin{Con}[\cite{Ilmanen1995}]
 Suppose that  ${M}_{0}  \subset {\mathbb{R}}^{n+1}$ is a smooth closed embedded hypersurface. A time-slice of any tangent flow of the MCF starting at 
  ${M}_{0}$ has a singular set of dimension at most $n-3$.
\end{Con}

\printbibliography[heading=subbibliography]
\end{refsection}

\newpage

\section{Gauss curvature flow}

\begin{refsection}

\begin{itemize}
\item The asymptotic shape of the `standard' Gauss curvature flows. There has been some recent progress on this, and some claim it is solved. A little controversial, but I think we can (and should) successfully clarify the issue.
\end{itemize}

\printbibliography[heading=subbibliography]
\end{refsection}

\section{Inverse mean curvature flow}

\begin{refsection}

\printbibliography[heading=subbibliography]
\end{refsection}

\section{Willmore flow}

\begin{refsection}

\begin{itemize}
\item Flow for surfaces with higher genus
\item Flow approach to duality theorem in higher codimension
\item Finite-time singularities for the flow
\item Weak solutions for the flow
\end{itemize}

\printbibliography[heading=subbibliography]
\end{refsection}

\section{Applications of extrinsic geometric flows}

\begin{refsection}
Note: added extrinsic to the title, unless we wish to talk mostly about Ricci flow, Calabi flow, and the conformal flow of metrics.

\begin{itemize}
\item Schoenflies, see HUisken-Sinestrari.
\item IMCF and Penrose? Overshadowed by the conformal flow, maybe not.
\item Isoperimetric? Geometric inequalities? Not sure there is a big thing that flows can do that e.g. measure theory and calculus of variations can't.
\item Entropy inequalities of Bernstein and co are a possibility?
\end{itemize}

\printbibliography[heading=subbibliography]
\end{refsection}

\section{List of symbols and abbreviations}


\end{document}





