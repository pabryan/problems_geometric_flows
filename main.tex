\documentclass{amsart}
\headheight=6.15pt \textheight=8.75in \textwidth=6.5in
\oddsidemargin=0in \evensidemargin=0in \topmargin=0in

\title[Problems in Geometric Flows and Solitons]{Problems in Geometric Flows and Solitons}
\author[]{Hojoo Lee} 
\author[]{Peter Wheeler} 
\author[]{around 8 more for initial draft}

\theoremstyle{definition}
\newtheorem{Thm}{Theorem}[section]
\newtheorem{Que}{Question}[section]
\newtheorem{Prob}{Problem}[section]
\newtheorem{Con}{Conjecture}[section]
\newtheorem{Cor}{Corollary}[section]
\newtheorem{Lem}{Lemma}[section] 
\newtheorem{Def}{Definition}[section] 
\newtheorem{Exa}{Example}[section] 
\newtheorem{Rem}{Remark}[section] 
\newtheorem{Prop}{Proposition}[section] 

\usepackage{epsfig,color}
 \usepackage{hyperref}
 
\begin{document}
\maketitle
\tableofcontents

\section{Introduction}
\label{sec:intro}
 
 \newpage
 
\section{Overture} 

\begin{enumerate}
\item[]
\item[] \textbf{\textcolor{black}{Notice for the arXiv version (updated on 15 May, 2017.)}}
\item[]
\item Part of this work will be edited into an article, which will be submitted to arXiv.
\item The tentative working title is \textbf{Open problems in geometric analysis}. 
\item {\textcolor{red}{The initiator of new section will be included as a co-author of the arXiv article.}} 
\item The list of contributors will be listed in the Acknowledgement.
\item {\textcolor{red}{Our target date to submit part of this work to math.DG is  \textbf{15 September 2017.}}} 
\item[]
\end{enumerate}
\begin{enumerate}
\item[] \textbf{\textcolor{black}{Instruction for Contributors (updated on 15 May, 2017.)}}
\item[]
\item {\textcolor{red}{The tentative deadline to initiate new section is \textbf{15 August 2017.}}}
\item Initiators of new sections are assumed to play a role of the editor of their sections.
\item \textcolor{red}{When contributors add known or improved results, please provide the \textbf{references}.} 
\item When you initiate a new section, please provide a background within three pages. 
\item  {\textcolor{red}{It is prohibited to add your own latex enviroment (like using \textit{newcommand}).}} 
\item Each section has the list of references at its end. Each item reveals the family name of the coauthors and the year of the publication.
For instance, use \textbf{Perelman 2002} to indicate his arXiv preprint \emph{The entropy formula for the Ricci flow and its geometric applications}.
\item[]
\item[] \textbf{\textcolor{black}{Editorial Board (updated on 15 May, 2017.)}}
\item[]
\item[] Paul Bryan  (\href{your mail}{your mail}) 
\item[] Hojoo Lee  (\href{momentmaplee@gmail.com}{momentmaplee@gmail.com}) 
\item[] Glen Wheeler  (\href{your mail}{your mail}) 
\item[]
\item[] Please do not hesitate to contact us if you have any comments, questions, and suggestions. 
\end{enumerate}

\section{Introduction to extrinsic curvature flows}
  
\newpage 

\section{Mean curvature flow}

\begin{itemize}
\item Estimation of the position of the extinction point -- see Bryant-Griffiths for a good discussion. Proved that there does not exist a formula from initial data, but a sharp estimate could be possible. An idea for this could be something related to the centre of mass.
\end{itemize}

\begin{Thm}[\cite{Gage Hamilton 1986}]
Under curve shortening flow, every simple closed convex curve in ${\mathbb{R}}^{2}$ remains convex and eventually becomes extinct in a round point.
\end{Thm}

\begin{Thm}[\cite{White 1995}]
Let ${M}_{t}$ be a weak MCF in ${\mathbb{R}}^{2}$ that starts from a closed surface ${M}_{0}$ of genus $g_0$. Then at each time $t > 0$ where ${M}_{t}$ is smooth, the genus of  ${M}_{t}$ is at most $g_0$.
\end{Thm}

\begin{Thm}[\cite{Huisken 1984}]
Under MCF, every closed convex hypersurface in ${\mathbb{R}}^{n+1 \geq 3}$ remains convex and eventually becomes extinct in a round point.
\end{Thm}
 
\begin{Thm}[\cite{White 2000}]
The singular set of a compact mean convex MCF in ${\mathbb{R}}^{n+1}$ has parabolic Hausdorff dimension at most $n-1$.
\end{Thm}

\begin{Thm}[\cite{Colding Minicozzi 2012}]
In ${\mathbb{R}}^{n+1}$ the round hypersphere ${\mathbb{S}}^{n}$ is the only closed smooth $F$-stable shrinker.
\end{Thm}

\begin{Thm}[\cite{Colding Ilmanen Minicozzi White 2013}] \label{Colding Ilmanen Minicozzi White 2013}
Given $n$, there exists  $\varepsilon = \varepsilon(n) > 0$ so that if $\Sigma \subset {\mathbb{R}}^{n+1}$
is a closed shrinker not equal to the round hypersphere, then $\lambda\left(\Sigma\right) \geq \lambda\left({\mathbb{S}}^{n}\right) + \varepsilon$. Moreover,
if $\lambda\left(\Sigma\right) \leq min \left\{  \lambda \left({\mathbb{S}}^{n-1}\right) , 3 \right\}$, then $\Sigma$ is diffeomorphic to ${\mathbb{S}}^{n}$.
\end{Thm}
 
 \begin{Thm}[\cite{Bernstein Wang 2016}]
Theorem \ref{Colding Ilmanen Minicozzi White 2013} holds  with $\varepsilon = 0$ for any closed hypersurface.
\end{Thm}

\begin{Con}[\cite{Colding Minicozzi Pedersen 2015}] Theorem \ref{Colding Ilmanen Minicozzi White 2013} holds for any 
non-flat shrinker ${\Sigma}^{n}  \subset {\mathbb{R}}^{n+1}$ with $n \leq 6$.
\end{Con}

\begin{Thm}[\textbf{Rigidity of cylinders}, \cite{Colding Ilmanen Minicozzi 2015}]
If a singular point of an MCF is cylindrical, then for every tangent flow there is a multiplicity one cylinder. 
\end{Thm}

\begin{Con}[\cite{Colding Minicozzi Pedersen 2015}] Let ${M}_{t}$  be an MCF flow of smooth closed hypersurfaces in 
${\mathbb{R}}^{n+1}$. If the flow has a cylindrical singularity at time t0 and at the point $x_0 \in {\mathbb{R}}^{n+1}$, then in an 
entire space-time neighborhood of $\left(x_0,t_0\right)$ the evolving hypersurfaces have positive mean curvature.
\end{Con}

\begin{Thm}[\textbf{Conical asymptotic rigidity}, \cite{Wang 2014}]
If ${\Sigma}_{1}$ and ${\Sigma}_{2}$ are shrinkers in ${\mathbb{R}}^{n+1} \ B_{R}$ that have boundary in $\partial B_{R}$ 
and are asymptotic to the same cone, then they coincide.
\end{Thm}

 \begin{Thm}[\cite{Colding Minicozzi 2015a}] Let ${M}_{t}$ be an MCF in ${\mathbb{R}}^{n+1}$. At each cylindrical singular 
 point the tangent flow is unique. That is, any other tangent flow is also a cylinder with the same ${\mathbb{R}}^{k}$ factor 
 that points in the same direction.
\end{Thm}
   
\begin{Con}[\cite{Colding Minicozzi 2016}]  
  Let ${M}_{t}$ be an MCF of closed embedded hypersurfaces in ${\mathbb{R}}^{n+1}$ with only cylindrical singularities. Then 
  the space-time singular set has only finitely many components.
\end{Con}

\begin{Con}[\cite{Ilmanen 1995}]  
 Suppose that  ${M}_{0}  \subset {\mathbb{R}}^{n+1}$ is a smooth closed embedded hypersurface. A time-slice of any tangent flow of the MCF starting at 
  ${M}_{0}$ has a singular set of dimension at most $n-3$.
\end{Con}

\newpage

\section{Gauss curvature flow}

\begin{itemize}
\item The asymptotic shape of the `standard' Gauss curvature flows. There has been some recent progress on this, and some claim it is solved. A little controversial, but I think we can (and should) successfully clarify the issue.
\end{itemize}
  
\section{Inverse mean curvature flow}

\section{Willmore flow}

\begin{itemize}
\item Flow for surfaces with higher genus
\item Flow approach to duality theorem in higher codimension
\item Finite-time singularities for the flow
\item Weak solutions for the flow
\end{itemize}
 
\section{Applications of extrinsic geometric flows}

Note: added extrinsic to the title, unless we wish to talk mostly about Ricci flow, Calabi flow, and the conformal flow of metrics.

\begin{itemize}
\item Schoenflies, see HUisken-Sinestrari.
\item IMCF and Penrose? Overshadowed by the conformal flow, maybe not.
\item Isoperimetric? Geometric inequalities? Not sure there is a big thing that flows can do that e.g. measure theory and calculus of variations can't.
\item Entropy inequalities of Bernstein and co are a possibility?
\end{itemize}

\section{List of symbols and abbreviations}


 \begin{thebibliography}{A}

\bibitem[Andrews 2012]{Andrews 2012} 
B. Andrews, {\emph{Noncollapsing in mean-convex mean curvature flow}}. 
Geometry \& Topology, 16 (2012), No. 3, 1413--1418.
 
\bibitem[Bernstein Wang 2016]{Bernstein Wang 2016} 
J. Bernstein and L. Wang, \emph{A sharp lower bound for the entropy of closed hypersurfaces up to dimension six}, 
206 (2016), No. 3, 601--627.

\bibitem[Brendle 2016]{Brendle 2016}
S. Brendle, \emph{Embedded self-similar shrinkers of genus $0$}, Annals of Math. 183, 715--728 (2016).

\bibitem[Calabi1958]{Calabi1958} 
E. Calabi, \emph{An extension of E. Hopf's maximum principle with an application to Riemannian geometry}.
Duke Math. J. 25 (1958), 45--56.

\bibitem[Colding Minicozzi 2012]{Colding Minicozzi 2012}
T. H. Colding and W. P. Minicozzi II,
\emph{Generic mean curvature flow I; generic singularities},  Annals of Math., Volume 175 (2012), No. 2, 755--833.

\bibitem[Colding Minicozzi 2015a]{Colding Minicozzi 2015a}
T. H. Colding and W. P. Minicozzi II, \emph{Uniqueness of blowups and \L{}ojasiewicz inequalities}, Annals of Math., 182 (1) (2015), 221--285.

\bibitem[Colding Minicozzi 2015b]{Colding Minicozzi 2015b}
T.H. Colding and W.P. Minicozzi II, \emph{\L{}ojasiewicz inequalities and applications}, Surveys in Differential Geometry, Vol. 19
Regularity and evolution of nonlinear equations
Essays dedicated to Richard Hamilton, Leon Simon, and Karen Uhlenbeck, International Press (2015), 63--82.

\bibitem[Colding Minicozzi 2016]{Colding Minicozzi 2016}
Tobias Holck Colding, and William P. Minicozzi II, \emph{The singular set of mean curvature flow with generic singularities}, 
Inventiones Math., 204 (2) (2016), 443--471.
 
 \bibitem[Colding Ilmanen Minicozzi 2015]{Colding Ilmanen Minicozzi 2015}
  T. H. Colding, T. Ilmanen, and W. P. Minicozzi II, {\emph{Rigidity of generic singularities of mean curvature flow}}, 
 Publications math\'{e}matiques de l'IH\'{E}S,  121 (2015), No. 1, 363--382.
 
 
\bibitem[Colding Minicozzi Pedersen 2015]{Colding Minicozzi Pedersen 2015}
T.H. Colding, W.P. Minicozzi II and E.K. Pedersen, {\emph{Mean curvature flow}}, Bulletin of the AMS, 52 (2015), No. 2, 297--333.


\bibitem[Colding Ilmanen Minicozzi White 2013]{Colding Ilmanen Minicozzi White 2013} 
Tobias Holck Colding, Tom Ilmanen, William P. Minicozzi II, and Brian White, 
{\emph{The round sphere minimizes entropy among closed self-shrinkers}}, J. Differential Geom. 95 (2013), No. 1, 53--69. 
 
\bibitem[Evans  Spruck 1991]{Evans  Spruck 1991} 
L. C. Evans and J. Spruck, \emph{Motion of level sets by mean curvature I}, J. Differential Geom.  33 (1991) 635--681.

\bibitem[Evans  Spruck 1992a]{Evans  Spruck 1992a} 
L. C. Evans and J. Spruck, 
\emph{Motion of level sets by mean curvature II}. Trans. Amer. Math. Soc. 330 (1992), No. 1, 321--332. 

\bibitem[Evans  Spruck 1992b]{Evans  Spruck 1992b} 
L.C. Evans and J. Spruck, 
\emph{Motion of level sets by mean curvature. III}. 
J. Geom. Anal. 2 (1992), No. 2, 121--150. 

\bibitem[Evans  Spruck 1995]{Evans  Spruck 1995} 
L.C. Evans and J. Spruck, 
\emph{Motion of level sets by mean curvature. IV}. J. Geom. Anal. 5 (1995), No. 1, 77--114. 

\bibitem[Gage Hamilton 1986]{Gage Hamilton 1986}
M. Gage and R. S. Hamilton, \emph{The heat equation shrinking convex plane curves}, J. Differential Geom. 
23 (1986), No. 1, 69--96.

\bibitem[Grayson 1989]{Grayson 1989}
M. Grayson,  \emph{Shortening embedded curves},  Ann. of Math. (2) 129 (1989), No. 1, 71--111. 

\bibitem[Huisken 1984]{Huisken 1984}
G. Huisken, Flow by mean curvature of convex surfaces into spheres, J. Differential Geom. 20 (1984), no. 1, 237--266.  

\bibitem[Huisken 1990]{Huisken 1990}
G. Huisken, \emph{Asymptotic behavior for singularities of the mean curvature flow}. J. Differential
Geom. 31 (1990), no. 1, 285--299.

\bibitem[Huisken 1993]{Huisken 1993} 
G. Huisken, \emph{Local and global behaviour of hypersurfaces moving by mean curvature}. Differential geometry:
partial differential equations on manifolds (Los Angeles, CA, 1990), 175Ð191, Proc. Sympos. Pure Math.,
54, Part 1, Amer. Math. Soc., Providence, RI, 1993.

\bibitem[Huisken Sinestrari 1999a]{Huisken Sinestrari 1999a}
G. Huisken and C. Sinestrari, 
\emph{Mean curvature flow singularities for mean convex surfaces}, 
Calc. Var. Partial Differential Equations, 8 (1999), 1--14.

\bibitem[Huisken Sinestrari 1999b]{Huisken Sinestrari 1999b}
G. Huisken and C. Sinestrari, 
{\emph{Convexity estimates for mean curvature flow and singularities of mean convex surfaces}},
 Acta Math. 183 (1999), no. 1, 45--70.
 
 \bibitem[Ilmanen 1995]{Ilmanen 1995}
 T. Ilmanen,  {\emph{Singularities of Mean Curvature Flow of Surfaces}}, preprint, 1995.

\bibitem[Wang 2014]{Wang 2014} 
 Lu Wang, {\emph{Uniqueness of self-similar shrinkers with asymptotically conical ends}}, J. Amer. Math. Soc. 27 (2014), no. 3, 613--638,
 
  
\bibitem[White 1995]{White 1995} 
B. White, \emph{The topology of hypersurfaces moving by mean curvature}, Comm. Anal.
Geom. 3 (1995), no. 1-2, 317--333.

\bibitem[White 2000]{White 2000} 
B. White, \emph{The size of the singular set in mean curvature flow of mean-convex sets}. 
J. Amer. Math. Soc. 13 (2000), no. 3, 665--695.

\bibitem[White 2003]{White 2003} 
B. White, \emph{The nature of singularities in mean curvature flow of mean-convex sets}.
J. Amer. Math. Soc. 16 (2003), no. 1, 123--138.


\end{thebibliography}

\end{document}





